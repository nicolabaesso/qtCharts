% Don't change anything in this top section!

\documentclass[10pt]{article} 

\usepackage{graphicx,gb4e,qtree,latexsym,tree-dvips,times,tipa,stmaryrd}
\usepackage[normalem]{ulem}
\usepackage[]{natbib}
\usepackage{hyperref}


\newcommand{\denplain}[1]{$\llbracket$#1$\rrbracket$} 

\newcommand{\denote}[2][]{$\llbracket$#2$\rrbracket$^{{\ensuremath #1}}} 

\newcommand{\ab}[1]{$\langle$#1$\rangle$}

\newcommand{\refp}[1]{(\protect\ref{#1})}

\newcommand{\itab}[1]{\hspace{0em}\rlap{#1}}
\newcommand{\tab}[1]{\hspace{.2\textwidth}\rlap{#1}}

\setlength{\parindent}{.5 in}
\setlength{\textwidth}{6.5 in}
\setlength{\oddsidemargin}{0 in}
\setlength{\textheight}{9.0 in}
\setlength{\topmargin}{-0.7 in}


\pagestyle{empty}

\author{Nicola Baesso}
\title{Relazione Progetto}

\begin{document}
	\setlength{\baselineskip}{13 pt}
	
	\maketitle
	
	% Here is the part of the .tex file that you have to worry about!
	
	29/11/2021 (2 ore): creazione progetto tramite l'utilizzo di video d tutor
	
	04/12/2021 (1 ora): creazione UML classi
	
	06/12/2021 (1 ora): proseguo UML
	
	18/12/2021 (1 ora): inizio stesura codice, header e classe "data". Next step: "dataHandler"
	
	20/12/2021 (1 ora): continuo stesura codice. Next step gestione file XML
	
	27/12/2021 (3 ore): stesura codice GUI, classe Chart per la visualizzazione, modifica di datahandler. Next step: implementazione della gestione file XML in DataHandler
	
	29/12/2021 (2 ore): continuazione di codice per fileHandler, creazione header per i grafici.
     
    30/12/2021 (3 ore): creazione della classe model per il collegamento di model e view, bug fixes, creazione di exception.
    
    03/01/2022 (3 ore): modifica uml, metodo di lettura dati da file. Inizio classe Controller
    
    04/01/2022 (6 ore): continuo UML, continuo controller.
    
    05/01/2022 (3 ore): continuo controller, classi Chart.
    
    07/01/2022 (6 ore): continuo view e charts
    
    10/01/2022 (2 ore): continuo view e controller
    
    11/01/2022 (1 ora): modifica scrittura di nuovo file (salvataggio e titolo chart)
    
    14/01/2022 (2 ore): modifica window. Da proseguire
    
    15/01/2022 (2 ore): proseguo modifica window. Da fare: connettere con controller (e model di conseguenza)
    
    16/01/2022 (1 ora): cambio di layout ad apertura di un nuovo file. Da fare: vedi 15/01/2022 e implementare anche nella creazone di un file
    
    17/01/2022 (3 ore): step su 15/01/2022, da finire. Risolto bug su BarChart
    
    18/01/2022 (4 ore): miglioramenti, da completare modifiche file. UPDATE: risolto modifiche file. Da fare: tema scuro, possibilità di modificare il nome del grafico, test...
    
    23/01/2022 (1 ora): modifica nome grafico e aggiunta icona applicazione. Da fare: tema scuro, auto-update, test...
    
    25/01/2022 (3 ore): sistemazione codice.
    
    \newpage
    \tableofcontents
    \newpage
    \section{Introduzione}
     Questo progetto riguarda un applicativo C++/Qt per la visione di dati sotto forma di grafici. L'applicazione permette una gestione ottimale dei dati, senza rinunciare alla comodità di un' interfaccia grafica.
    \subsection{Strumenti di sviluppo utilizzati}
    Il progetto è stato scritto utilizzando l'IDE QtCreator alla versione 4.5.2, ed è stato utilizzata la versione 5.9.5 del framework Qt. 
    La seguente relazione è stata create tramite LaTex utilizzando TexStudio.
    Il codice è stato scritto e testato in Windows 10 (versione 21H1) e in Debian 11 Bullseye.
    Inoltre, il codice è stato versionato tramite Git ed è anche disponibile (sotto il nome di QtCharts) a \href{https://github.com/nicolabaesso/qtCharts}{questo indirizzo}.
    \subsection{Compilazione ed esecuzione in sistemi Linux}
    Per la compilazione "manuale" di questo progetto, sono necessari i seguenti kit:\\
    -qt5-default (Qt 5.9.5)\\
    -libqt5charts5-dev (QtCharts)
    \\
    \\
    Inoltre, è fondamentale utilizzare il file Progetto.pro, fornito assieme a tutti i sorgenti necessari alla compilazione.\\
    \\
    La sequenza dei comandi è come segue:\\
    -qmake Progetto.pro\\
    -make
    \\
    Alla fine verrà generato un file chiamato Progetto, lo si può eseguire scrivendo ./Progetto\\\\
    L'utilizzo di un file .pro differente (tra cui quello generato da qmake -project) o il mancato utilizzo dei kit riportati non assicura la corretta compilazione del programma.
    \section{Funzionamento dell'applicazione}
    \subsection{Avvio e schermata iniziale}
    All'avvio, si verrà accolti dalla seguente schermata:
    
    insert image here
    
    L'applicazione genera un file di esempio, contenente 5 valori, al fine di mostrare sin dal principio le funzionalità presenti.
     
    \subsection{Interfaccia Grafica}
    L'interfaccia può essere suddivisa in 3 aree principali:\\
    -la toolbar, con i tasti necessari alla creazione di un nuovo file e al salvataggio di un file, nonchè a cambiare la tipologia di grafico visualizzato e al titolo dello stesso\\
    -la sezione "dati", che contiene i tasti per l'aggiunta di un nuovo dato, il salvataggio dei dati visibili e l'eliminazione di un determinato dato, oltre ai dati stessi con la possibilità di modifica\\
    -la sezione "grafici", che contiene la visualizzazione dei dati sotto forma del grafico scelto, costantemente aggiornato ad ogni modifica dei dati
    
    \section{Tempistiche(Ore di lavoro)}
    \section{Caratteristiche tecniche}
    \subsection{URL e classi coinvolte}
    insert image here\\
    Al fine di evitare l'utilizzo di codice non estensibile, si è optato per l'utilizzo del modello MCV, che utilizza tre elementi (Model, View e Controller) per una migliore gestione degli eventi. Inoltre, come spiegato in maniera dettagliata successivamente, ogni classe presente nell'applicativo ha un ruolo ben preciso nella funzionalità.
    \subsubsection{Classe Data}
    La classe Data rappresenta un singolo dato letto da file, o generato automaticamente all' avvio (si veda la sezione "Avvio e schermata iniziale" per maggiori dettagli).
    Tale classe è composta da una stringa, che rappresenta una breve descrizione del dato, e un valore numerico di tipo double, che rappresenta il valore del dato. Sebbene per molte circostanze un valore di tipo double sia eccessivo, si è ritenuto adatto a rappresentare la maggior parte dei dati possibili.
    \subsubsection{Classe DataHandler}
    La classe DataHandler rappresenta i dati presenti nel file letto per ultimo. Inoltre si occupa anche della modifica e della cancellazione dei dati stessi.
    La classe è composta da una stringa, che rappresenta il titolo descrittivo dei dati, nonchè la stringa che verrà visualizzata come titolo del grafico, e una struttura dati, più precisamente un vector, contente elementi di tipo Data, che rappresenta i dati letti inizialmente dal file.
    Si è optato per l'utilizzo della classe Vector, e non per altre strutture dati, magari semplici come un array, poichè le funzioni offerte da suddetta classe aiutano a gestire meglio i dati.
    \subsubsection{Classe FileHandler}
    La classe FileHandler ha lo scopo di lettura vera e propria dei dati dal file, nonchè del salvataggio di essi nello stesso o in un nuovo file. Contiene un campo QString, che rappresenta il nome del file, e un DataHandler che rappresenta i dati letti o da salvare nel file.
    \subsubsection{Classe Model}
    La classe Model è la classe che gestisce ogni singolo aspetto del modello. Essa contiene un campo FileHandler, che rappresenta  il file in lettura/scrittura.
    \subsubsection{Classe Controller}
    La classe Controller funge da tramite tra model e view. In particolare, gestisce gli eventi evocati nella view tramite interfaccia grafica, e li collega agli eventi nel model. I suoi unici campi sono di tipo Model* e Window*, ovvero puntatori a, rispettivamente, modello e view. Inoltre, ha principalmente slot invocati dai signal che emette la classe Window.
    \subsubsection{Classe Window}
    La classe Window si occupa di ogni particolare relativo all'interfaccia grafica. In altre parole, Si occupa della visualizzazione dei dati e dei grafici, nonchè alla creazione e visualizzazione dei vari menù e finestre. Contiene svariati campi, tra cui di tipo QPushButton*, QMenu*,QDialog*...\\Inoltre, ha un campo di tipo Controller* per associare ogni elemento interagibile ad un determinato segnale.
    \subsubsection{Classe Chart}
    \subsubsection{Classe BarChart}
    \subsubsection{Classe LineChart}
    \subsubsection{Classe PieChart}
    \subsection{Gerarchie di tipi}
    \subsection{Chiamate polimorfe}
    \subsection{Formato di file usato per I/O di file}
    Per questo progetto si è scelto di utilizzare il formato XML per la memorizzazione dei dati in file. Sebbene sia un formato "vecchio", permette una gestione più ordinata delle informazioni e permette di visualizzarle da un qualunque editor di testo.
    Per questo motivo, i file letti devono obbligatoriamente avere la formattazione visualizzabile anche da questa immagine: insert image here
	% That's it!
	
\end{document}